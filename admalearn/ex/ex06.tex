\documentclass[12pt,a4paper]{article}
%%%%%%%%%%%%%%%%%%%%%%%%%%%%%%%%%%%%%%%%%%%
% write which packages is used in your text
\usepackage{amsmath}
\usepackage[amsmath]{ntheorem}
\usepackage{lastpage}
\usepackage{amsfonts}
\usepackage{amssymb}
\usepackage{graphicx}
\usepackage{hyperref}

%now you can write a text with French accents 
\usepackage[utf8]{inputenc}

%%%%%%%%%%%%%%%%%%%%%%%%%%%%%%%%%%%%%%%%%%%
% write your abbreviation of symbols here

%fill the brackets with your firstname and lastname
\def\StudentName{FirstName LastName} 
%fill the brackets with your matricule (poly student number)
\def\StudentMatricule{StudentNumber}
% exercise of which week? 
\def\ExerciseNo{6}




\def \y {\mathbf y}
\def \X {\mathbf X}
\def \A {\mathbf A}
\def \t {^\top}
\def \inv {^ {-1}}
\def \x {\mathbf x}
\def \bbeta {\boldsymbol \beta}
\def \SSigma {\boldsymbol \Sigma}
\def \eeps {\boldsymbol \varepsilon}

\def \Q {\mathbf Q}
\def \R {\mathbf R}
\def \q {\mathbf q}
\def \zero {\mathbf 0}

\def \L {\mathbf L}
\def \U {\mathbf U}

\def \A {\mathbf A}
\def \P {\mathbf P}

\def \D {\mathbf D}
\def \MSE {\mathrm{MSE}}
\def \E {\mathbb{E}}
\def \V {\mathbb{V}}

\def \sumi {\sum_{i=1}^n}
\def \sumj {\sum_{j=1}^p}

\def \argmin {\mathrm {argmin}~}
\def \argmax {\mathrm {argmax}~}
\def \sign {\mathrm {sign}}
\def \N {\mathcal{N}}
\def \eye {\mathbf{I}}


%%%%%%%%%%%%%%%%%%%%%%%%%%%%%%%%%%%%%%%%%%%


% define your tags here
\theoremheaderfont{\normalfont\bfseries}
\theorembodyfont{\normalfont}
\newtheorem{exercise}{Exercise}
\numberwithin{exercise}{section} % important bit

\newtheorem{solution}{Solution}
\numberwithin{solution}{section} % important bit
%%%%%%%%%%%%%%%%%%%%%%%%%%%%%%%%%%%%%%%%%%%

% define what to write in paper margins
\usepackage{fancyhdr}
\pagestyle{fancy}
\lhead{{\bf~\StudentName,~\StudentMatricule}}
\chead{}
\rhead{\emph{Exercise no~\ExerciseNo}}
\lfoot{}
\cfoot{\thepage~/~\pageref{LastPage}}
\rfoot{}
%%%%%%%%%%%%%%%%%%%%%%%%%%%%%%%%%%%%%%%%%%%


% gives more space and expands margins
\textwidth 6.4in
\textheight 9in \oddsidemargin 0in \evensidemargin 0in \topmargin -0.5in
\renewcommand{\baselinestretch}{2} 
% this puts more space between lines, so that I can write comments in between
%%%%%%%%%%%%%%%%%%%%%%%%%%%%%%%%%%%%%%%%%%%

\begin{document}


%%%%%Generates your first page
\begin{titlepage}
\begin{center}
\textsc{\LARGE Exercise no \ExerciseNo}\\[1.5cm]
\vspace{2in}
\textsc{\Large \StudentName~\StudentMatricule}\\[0.5cm]
\textsc{Statistical Machine Learning}\\[0.5cm]
\today
\end{center}
\end{titlepage}
%%%% Title Page ends here



\section{Mathematical Statistics}
\begin{exercise}
Show that the kernel smoothing (weighted average)  is the solution of the following optimization if $f_\theta(x)=\theta_0$
 $$\hat \theta(x_0)= \mathrm{argmin}_\theta \sum_{i=1}^N K(x_0,x_i) \{y_i-f_\theta(x_i)\}^2,$$
 $$ \hat f(x_0)= f_{\hat \theta} (x_0)$$
\begin{itemize}
\item Find the link between this optimization problem and the  weighted linear regression. 
\item Find the solution of $\hat f$ for $f_\theta(x)=\theta_0+\theta_1 x$?
\item Find the solution of $\hat f$ for $f_\theta(x)=\theta_0+\sum_{j=1}^M \theta_j x^j$?
\end{itemize}
\end{exercise}
\begin{solution}
\end{solution}

\newpage

\section{Computation}
\begin{exercise}
Take $lpsa$ as the response variable $y$ and $lcp$ as the dependent variable $x$ for the prostate cancer data set\\
\url{https://web.stanford.edu/~hastie/ElemStatLearn/datasets/prostate.data}\\
Use polynomial order $M\in\{0,1,2,3\}$ local regression of Exercise 1.1  for different values of $M$, different kernels, different bandwidth $\lambda$ on prostate cancer data. Which $M$, which kernel, which bandwidth $\lambda$ do you prefer?\\
Use Rbf kernel, and apply $10$-fold cross-validation to tune $\lambda$  for this data. 
Compare different fits as order $M\in\{0,1,2,3\}$ changes in a graph.
\end{exercise}

\begin{solution}
Put your graph here.
\end{solution}
\newpage 

\section{Application}
\begin{exercise}
\begin{itemize}
\item Use $k$-nearest neighbours classifier and tune $k$ using 10-fold cross-validation  on the zip data to recognize digit $3$ from digit $8$. All digits $\{0,\ldots, 9 \}$ are available here.\\
\url{https://web.stanford.edu/~hastie/ElemStatLearn/datasets/zip.train.gz}
\item predict accuracy of the test set \\
\url{https://web.stanford.edu/~hastie/ElemStatLearn/datasets/zip.test.gz}
\end{itemize}
\end{exercise}
\begin{solution}
Put the \emph{confusion matrix} of the test set here.
\end{solution}





\end{document}