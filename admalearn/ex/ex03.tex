\documentclass[12pt,a4paper]{article}
%%%%%%%%%%%%%%%%%%%%%%%%%%%%%%%%%%%%%%%%%%%
% write which packages is used in your text
\usepackage{amsmath}
\usepackage[amsmath]{ntheorem}
\usepackage{lastpage}
\usepackage{amsfonts}
\usepackage{amssymb}
\usepackage{graphicx}
\usepackage{hyperref}

%now you can write a text with French accents 
\usepackage[utf8]{inputenc}

%%%%%%%%%%%%%%%%%%%%%%%%%%%%%%%%%%%%%%%%%%%
% write your abbreviation of symbols here

%fill the brackets with your firstname and lastname
\def\StudentName{FirstName LastName} 
%fill the brackets with your matricule (poly student number)
\def\StudentMatricule{StudentNumber}
% exercise of which week? 
\def\ExerciseNo{3}




\def \y {\mathbf y}
\def \X {\mathbf X}
\def \A {\mathbf A}
\def \t {^\top}
\def \inv {^ {-1}}
\def \x {\mathbf x}
\def \bbeta {\boldsymbol \beta}
\def \eeps {\boldsymbol \varepsilon}

\def \Q {\mathbf Q}
\def \R {\mathbf R}
\def \q {\mathbf q}
\def \zero {\mathbf 0}

\def \L {\mathbf L}
\def \U {\mathbf U}

\def \A {\mathbf A}
\def \P {\mathbf P}

\def \D {\mathbf D}
\def \MSE {\mathrm{MSE}}
\def \E {\mathbb{E}}
\def \V {\mathbb{V}}

\def \sumi {\sum_{i=1}^n}
\def \sumj {\sum_{j=1}^p}

\def \argmin {\mathrm {argmin}~}
\def \sign {\mathrm {sign}}
\def \N {\mathcal{N}}
\def \eye {\mathbf{I}}


%%%%%%%%%%%%%%%%%%%%%%%%%%%%%%%%%%%%%%%%%%%


% define your tags here
\theoremheaderfont{\normalfont\bfseries}
\theorembodyfont{\normalfont}
\newtheorem{exercise}{Exercise}
\numberwithin{exercise}{section} % important bit

\newtheorem{solution}{Solution}
\numberwithin{solution}{section} % important bit
%%%%%%%%%%%%%%%%%%%%%%%%%%%%%%%%%%%%%%%%%%%

% define what to write in paper margins
\usepackage{fancyhdr}
\pagestyle{fancy}
\lhead{{\bf~\StudentName,~\StudentMatricule}}
\chead{}
\rhead{\emph{Exercise no~\ExerciseNo}}
\lfoot{}
\cfoot{\thepage~/~\pageref{LastPage}}
\rfoot{}
%%%%%%%%%%%%%%%%%%%%%%%%%%%%%%%%%%%%%%%%%%%


% gives more space and expands margins
\textwidth 6.4in
\textheight 9in \oddsidemargin 0in \evensidemargin 0in \topmargin -0.5in
\renewcommand{\baselinestretch}{2} 
% this puts more space between lines, so that I can write comments in between
%%%%%%%%%%%%%%%%%%%%%%%%%%%%%%%%%%%%%%%%%%%

\begin{document}


%%%%%Generates your first page
\begin{titlepage}
\begin{center}
\textsc{\LARGE Exercise no \ExerciseNo}\\[1.5cm]
\vspace{2in}
\textsc{\Large \StudentName~\StudentMatricule}\\[0.5cm]
\textsc{Statistical Machine Learning}\\[0.5cm]
\today
\end{center}
\end{titlepage}
%%%% Title Page ends here



\section{Optimization}
\begin{exercise}
Find the solution of 
\begin{itemize}
\item Orthogonal ridge: $S(\bbeta) = {1\over 2n} \sumi \sumj (y_{ij}-\beta_j)^2 + {\lambda} \sumj \beta_j^2 $ \\
Hint: first solve $S_j= {1\over 2n} \sumi (y_{ij}-\beta_j)^2 + {\lambda} \sumj | \beta_j |$ , and then generalize. 

\item Orthogonal lasso: $S(\bbeta) = {1\over 2n} \sumi \sumj (y_{ij}-\beta_j)^2 + {\lambda } \sumj | \beta_j | $ \\
Hint: first solve $S_j= {1\over 2n} \sumi (y_{ij}-\beta_j)^2 + {\lambda } | \beta_j |$ . Solve once for $\beta_j \geq0$ and another time for $\beta_j <0$. Put the pieces together. Then generalize. 

\item Ridge regression: minimize
$$S(\bbeta) = {1\over 2} (\y - \X\bbeta)\t (\y-\X\bbeta) + {\lambda} \bbeta\t\bbeta$$
Hint: use vector differentiation formulas of lecture01.
\item Lasso: Write the coordinate descent algorithm for minimizing
$$S(\bbeta) = {1\over 2} (\y - \X\bbeta)\t (\y-\X\bbeta) + {\lambda} \sumj |\beta_j|$$

\end{itemize}
\end{exercise}
\begin{solution}
\end{solution}

\newpage

\begin{exercise}
This extension of least squares allow you to model closed surfaces such as 3D scan of body or face.  \\

Find the linearly constrained least squares estimator $(\y - \X\bbeta)\t(\y-\X\bbeta)$ subject to 
$\mathbf T\bbeta=\mathbf b$ in which $\mathbf T$ and $\mathbf b$ both are known. Hint: use the Lagrangian dual. \\
How do you compute this estimator efficiently? 
\end{exercise}
\begin{solution}
\end{solution}


\begin{exercise}
Show that the eigen values of $\mathbf A+\lambda \mathbf I$ equals $\lambda_i+\lambda$ where $\lambda_i$'s are the eigenvalues of $\mathbf A$. Use this result to argue that the ridge regression improves the condition number of $\X\t\X$.\\
\end{exercise}
\begin{solution}
\end{solution}
\newpage 
\section{Mathematical Statistics}
\begin{exercise}
Show the kernel density estimator $\hat f (y)={1\over n\lambda} \sum_{i=1}^n K({y_i-y\over \lambda })$ 
 is a probability density,  for any non-negative kernel that $ \int_{-\infty}^\infty K(y)dy=1$, \\
 Hint : you must show $\hat f(y)\geq 0$ and $\int_{-\infty}^\infty \hat f(y)dy=1$.\\
\end{exercise}


\begin{exercise}
A weighted linear model with weights $ \mathbf W$ is called ordinary linear regression if $\mathbf 
W=\sigma^2 \mathbf I$ for a known $\sigma^2$.
\begin{itemize}
\item Find the maximum likelihood estimator of $ \bbeta$ for the weighted linear regression. Weighted linear regression   is $\mathbf y=\X \bbeta+ \eeps$ while $\boldsymbol \varepsilon \sim N(\mathbf 0, \mathbf W)$ and $\mathbf W_{n\times n}$ is the known variance covariance matrix of $\boldsymbol \varepsilon$. 
\item How do you compute the wighted least squares efficiently?
\item What is $\mathbf W$ in $S(\bbeta) = \sumi a_i (y_i -\x_i\t \bbeta)^2 ?$
\end{itemize}
\end{exercise}

\newpage
\section{Implementation}

\begin{exercise}
How do you fit a weighted linear regression using a code that only fits the ordinary linear regression?\\
Hint: re-define $\y$  and $\X$ as a function of $\mathbf W$
\end{exercise}

\begin{exercise}
How do you fit ridge regression using a code that only fits the ordinary linear regression?\\ 
Hint: you must re-write the ridge estimator as a least squares problem and redefine a new (perhaps larger) $\y_\lambda$ and $\X_\lambda$ so that 
$(\X\t\X+\lambda \mathbf I)\inv \X\t \y = (\X_\lambda\t\X_\lambda)\inv \X_\lambda\t \y_\lambda$.
\end{exercise}
\newpage

\begin{exercise}
Chicago is well-known to be a crime city. Use R to discover where  to  buy a house in Chicago.\\ 
The crime data of Chicago are publicly available here \\
{\small
\url{https://data.cityofchicago.org/Public-Safety/Crimes-2001-to-present/ijzp-q8t2/data}} \\
 Take the subset of crimes committed in 2017 and plot a red dot (with enough transparency) for each \emph{theft} appeared during 2017 on googlemap. \\
Hint: Use \texttt{get\_map}, and \texttt{ggmap} functions from 
\texttt{ggmap} library to download the google map with appropriate zoom, longitude, and latitude. 
Then add red points using \texttt{geom\_point} function of \texttt{ggplot2} library with transparency on your Chicago map.  Using \texttt{ggplot} functions is not straightforward, check some simple examples first, before trying to overlay your points on the map.\\
Try to reproduce something similar to my graph (below).\\
\includegraphics[width=0.5\textwidth]{gmap}
\begin{solution}
Put your graph here.
\end{solution}
\end{exercise}


















\end{document}