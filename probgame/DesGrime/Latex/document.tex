%%This is a very basic article template.
%%There is just one section and two subsections.
\documentclass{article}
\usepackage{mathtools}
\usepackage{amssymb}


\usepackage{graphicx}

\begin{document}

\section{Nontransitive dice explained}

\subsection{Preface}
    
    The nontransitive dice discussed in this article can be viewed as a way to
    caricature complex interaction that we have to deal with daily. To gain
    understanding of the concepts behind those dice, an aspect of discrete
    mathematics, the binary relation, will be presented. While beign very
    high-level and abstract, it can be applied to a lot of concrete situation in
    almost every domain and it will help introduce transitivity and intransivity
    as a property of a relation. Briefs example of transitivity,
    antitransitivity andintransitive will be given, followed by an introduction
    to nontransitive dice, which is an applied example of an intransitive
    relation. Various situation using those dices will be
     analyzed mathematically and real life examples will be given.
	
	\subsubsection{Overview of relation}
    The relation is an abstract concept which links numbers of elements togheter.
    Consider a group of student and books of the library. We will define the
    relation $ \mathfrak{R} $ as every pair of students and books suchs as the
    student $A_{i}$ have read the book $B_{j}$.
    Formally, it looks like this :
	\begin{equation}
	\begin{split}
	
	\mathfrak{R} = (A,B, \text{have read})\\
	\text{which yeilds the pairs} \\
	(A_{i}, B_{j}) i,j \in \mathbb{R}
	
	\end{split}
	\end{equation}

    GRAPHICS HERE\\	
    
    We can also define relations inside only one group, suchs as a student-student
    relation.
    For example, lets consider the following :
	
	\begin{equation}
	\begin{split}
	
	\mathfrak{R} = (A,A, \text{know})\\
	\text{which yeilds the pairs} \\
	(A_{i}, A_{j}) i,j \in \mathbb{R}
	
	\end{split}
	\end{equation}

    GRAPHICS HERE\\	
    
	Thus, a relation from a set against a set is valid and make sens. This will be used to help define transitivity.
    
	\subsubsection{Transitivity from a mathematical point of view}
	\label{transitivity}
	In his book \underline{\textit{Mathematiques discretes}}, Kenneth H. Rosen defines the conditions
	for a relation to be transitive as follow : \\
	
    Let A,B,C be elements of a set S. \\
	If (A,B) \& (B,C) are valid transitions in the relation R,\\
    it implies that transition (A,C) exists. \\
    
    This definition is the stronger version of antitransitivity. The differences between the two 
    is that a intransitive relation needs all elements to be not transitive, as an antitransitive
     is qualified as such with 
    only one element or more not being transitive.\\
	GRAPHS HERE \\
	
	\subsubsection{Transitivity from an everyday life point of view}
    Transitive, intransitive and antitransitive situation are part of our
    everyday life. For a concrete example of transitive relation, lets consider
    the following example : The relation is characterized by ''Is in the same
    family as''. If you are in the same family as your father, good chances
    are that you are from the same family as your father's father. Now for the
    intransitive relation, let's consider the characterization of the relation
    as ''is father of'' This relation is intransitive because your father's
    father cannot be in any ways your father (hopefully), which follows the
    properties defined before for intransitivity. For the antitransitive
    relation, we could consider a group of people and ''is friend with'' as
    characterization. If in this group, A is friend with B and B with C,
    sometimes A is friends with C, sometimes not. Transitivity is not always
    applied there, therefore it is antitransitive.
    
    \subsection{Nontransitive dice}
    \label{NTDice}

    The nontransitive dice were invented as a game to illustrate intransitivity.
    In this game, the first player offers the opponent to choose a dice from a
    set of different dices. After that, the first player now pick his dice.
    After an arbitrary but fixed number of rolls, the player who won the most
    rolls win. This could potentially be used as a betting game, but not one of
    the fairest. This is due to the fact that nontransitive dice exhibits
    intransitivity as we will see later. First, we will be presented a method to
    evaluate the odds of the set of intransitive dices presented at
    \ref{fig:3NTDice}. After that, we will map the relation between those dices
    to exhibits intransitivity.
    
	\subsubsection{Comparing the dices}

    First, a legitimate question would be ''how can dices that have the same
    value from the sum of each faces present different odds one from another?''.
    A unsatisfactory answer would be to use intuition while looking at the
    dices face. Comparing the red and green one, for example, will reveal that
    the red dice can only win in the cases where the roll exhibits the red dice
    with 9 and the green with 0 or 5 ((R:9,G:0) , (R:9, G:5)). Since the red
    dice have only one chance in 6 to obtain 9, and the green 5 out of 6 to
    obtain anything but 0, we have a strong (and correct) feeling that the green
    dice \textit{weight} more than the red. The use of the word weight here is
    to illustrate that the green win over the red since it is \textit{heavier}.
    
    \subsection{Evaluating the odds}
    
    We saw in the previous section that the green dice seemed heavier than
    the red one. There's a mathematical way to prove this, and it is quite simple
    assuming a little knowledge of combinatorial probability. First of all, our
    dice are 6 faced ones. Hence, it exists $ 6 * 6 = 36$ possibility for one
    roll. \ref{fig:RvsG_1D} illustrates the odds of that roll. 
    
    %ADD GRAPH ROUGE-VERT-1D%
    
    The tree here represent the possibility associated for each faces. For
    example, with the green dice, we have 1/6 chances to get 0 and 5/6 chances
    for a 5. Likewise, for the red section, if we have obtained 0 from green,
    there's 1/6 chance to obtain 9 and 5/6 chance to obtain 4 for the red dice.
    We calculate the probability of getting a 0 from green and a 9 from red
    as $\tfrac{1}{6}*\tfrac{1}{6}$, the probability for (G:0,R:4) as
    $\tfrac{1}{6}*\tfrac{5}{6}$ and so forth. In the three, we identified the
    winning dice with the probability number with the color of the winning dice.
    
    While this confirms our taught that the green is \textit{heavier} than the
    red dice, it also proves that two different dices with the same value for
    the sum of their faces dosent necesserly have the same weight. This dosen't
    mean that the red one will always loose against the green at each rolls, but
    if the number of rolls is high enough, the probability shown in
    \ref{fig:RvsG_1D} and the rolls outcome will certainly have a similar look.
    
    Now lets compare the green dice against the blue one, as
    shown in \ref{fig:BvsG_1D}
    
    %ADD GRAPH BLEU-VERT-1D%
   
    Here we see that the blue dice win over the green one, therefore imnplying
    that the blue one is \textit{heavier} than the green one. This could also
    mean, with some simple deduction, that the blue is heavier than the red,
    since the green is heavier than the latter. Let's prove this wrong with
    \ref{fig:RvsB_1D}.
    
    %ADD GRAPH ROUGE-BLEU-1D%
    
    While this clearly shows that the red dice win over the blue, this also mean
    that the dice weight cannot be calculated globally. It must be calculated
    against each dice, because as we seen, blue wins over green, green wins over
    red but blue looses against red. 
    
	\subsubsection{Intransitivity in nontransitive dices set}

    While the previous graphs demonstrated the relation between the dice, we
    could apply the definition from \ref{sec:transitivity} to characterize the
    relation between the dices. If the three dices were normal dice (the 1 to 6
    dices), each one would have the same weight. Now, as the red weight more
    than the blue, the blue than the green and the green than the red, this
    exhibits the following relation :
    
    % ADD GRAPH RELATION_3D_1D %
    
    \ref{fig:RELATION_3D_1D} shows that the following relation exhibits an
    intransitive relation. In truns, if your opponents pick the dice first and
    you know the relation behind the dices, and the number of rolls is
    sufficiently large (about 20 rolls start to exhibits the dices properties),
    youre chances of winning will be a lot larger.

\subsection{Grime dice}
    
    The dices from \ref{fig:3NTDice} we studied at section \ref{sec:NTDice} was
    a subset of a larger set known as "Grime dice''. This larger sets exhibits
    intransivity in about the same way than the three previous dice, exept that
    Grime dices contains 5 different dices. This set is presented at the figure
    below :
    
    % ADD GRAPH GRIMEDICE-1D %
    
    \subsubsection{The intransitivity in Grime dice presented}
    
    As we already know, the red, blue and green dice already form a set of
    intransitive dices. We see from the figure below that they still form a
    subset inside the Grime dice set :
    
    We wont go in the details about how one dice beats another. It's up to the
    reader to apply the same method presented at \ref{sec:NTDice} to mesaure the
    dices weight against each others. 
    
    We see that adding 2 more dices adds some pair of dices in the relation.
    Now, every dices beats 2 dices and is beaten by two. Choosing the right dice
    after your two opponent's have chosen thiers can makes the odds beats the
    two of them.
    
    \subsubsection{A game in a game : using a pair of dice}
    
    Grime dices hold a pretty secret. You can have two different games 
    if you decide to change the rules. Choosing a pair of two identical dices
    and comparing the results of the sums of the pairs yeilds another
    intransitive relation similar to the first with different probabilities.
    This time, it is harder to guess the weight of the dices as we did before,
    so we go directly with a graph three that sums up the probabilities of the
    red dice against the pink one. 
    
    % ADD GRAPH ROUGE-ROSE-1D %
    % ADD GRAPH ROUGE-ROSE-2D %
    
    We can see that the odds have changed from one game to another. The relation
    diagram representing the new intransitive relation is as following :
    
    % ADD GRAPH GRIMEDICE-2D %
    
    Note however that even if the red dice beats the green one in this game,
    they almost weight the same. This means the number of rolls to beat the
    green will need to be really high. 
    
\subsection{Creating nontransitive dice}
    
    Looking back at Grime dices may makes us think that the reversing order of
    the relation depending on the use a single or a pair of dice is a natural
    property.
    It is not, and this behaviour is due to carefully chosen value on the dices.
    This section will explore how we can create such set of dice for three
    six-faced dice.
    
\subsection{Conclusion}
    Intransitivity and transitivity is a concept that benefits to be applied to as much situations as possible.
    To make the right choice in an entreprise, the manager gain to identify all of the pro's and con's of each, and 
    how it will transition in the buisness.	
\end{document}
